\chapter{Bevezetés}

A modern informatikai rendszerek mindennapi életünk szinte minden területén jelen vannak: okostelefonjainktól kezdve az online bankoláson át az ipari vezérlőrendszerekig. Ezeknek a rendszereknek az alapját jellemzően valamilyen operációs rendszer adja, amelynek feladata nemcsak az erőforrások hatékony kezelése, hanem a felhasználók és az adatok védelme is. Az operációs rendszerek biztonsága ezért kiemelt fontosságú kérdés – egy-egy hibából adódó sérülékenység akár milliók adatait, vagy kritikus infrastruktúrák működését is veszélyeztetheti.

Az utóbbi években a támadási felület jelentősen megnőtt: egyre összetettebb szoftverek, konténerizáció, felhőszolgáltatások és mikro-szolgáltatásos architektúrák épülnek egymásra. Miközben a fejlesztők gyakran gyorsaságra és funkcionalitásra optimalizálnak, a biztonsági szempontok sokszor háttérbe szorulnak, vagy csak később kerülnek előtérbe. Ezzel párhuzamosan megjelentek és folyamatosan fejlődnek azok a kernel-szintű védelmi mechanizmusok, amelyek célja a folyamatok jogosultságainak finomhangolása és a lehetséges károk korlátozása kompromittálódás esetén. Ilyen eszközök például a Linux rendszermaghoz kapcsolódó biztonsági modulok és sandbox-megoldások.

E dolgozat központi témája, hogy miként járulnak hozzá a Landlock, a BPF-alapú megközelítések és a hasonló mechanizmusok az operációs rendszerek biztonságának növeléséhez. A dolgozat áttekinti továbbá az olyan, széles körben ismert megoldások szerepét is, mint a SELinux és az AppArmor. A fókusz nem kizárólag az elméleti működés ismertetésén van, hanem azon is, hogy gyakorlati példákon keresztül bemutassa: valós alkalmazásokban, valós sérülékenységek esetén milyen védelmi képességekkel rendelkeznek ezek az eszközök.

A dolgozat egyik fontos célja, hogy rámutasson: számos, a múltban kihasznált sérülékenység hatása mérsékelhető, vagy akár teljesen megelőzhető lett volna jól megtervezett sandboxolással és megfelelően konfigurált biztonsági mechanizmusokkal. Ennek érdekében a dolgozat konkrét eseteket vizsgál meg, és azt elemzi, hogy a bemutatott technológiák (például a Landlock által biztosított jogosultság-szűkítés) hogyan tudták volna korlátozni a támadók mozgásterét. Ez a visszatekintő megközelítés segít abban, hogy a különböző mechanizmusok gyakorlati hasznát ne csak elméleti szinten, hanem kézzelfogható példák alapján is értékelni lehessen.

A vizsgálat során nemcsak az egyes technológiák képességei, hanem a fejlesztői szemszög is központi szerepet kap. A biztonsági megoldások használata gyakran bonyolult, nehezen átlátható API-kon, illetve összetett konfigurációs szabályokon keresztül történik, ami visszatartja a fejlesztőket attól, hogy széles körben alkalmazzák ezeket. A dolgozat ezért egy saját fejlesztésű könyvtárat is bemutat, amelynek API-ja kifejezetten az egyszerű használhatóságot és a fejlesztőbarát megközelítést helyezi előtérbe. A cél az, hogy a biztonságosabb szoftverfejlesztés ne extra terhet, hanem jól integrálható, kényelmesen használható eszközt jelentsen.

A téma időszerűségét az is alátámasztja, hogy a kernel-szintű biztonsági mechanizmusok folyamatosan fejlődnek, új funkciókkal bővülnek, és egyre több disztribúcióban válnak alapértelmezetté vagy könnyen elérhetővé. Ugyanakkor egy ilyen gyorsan változó és technikailag összetett terület áttekintése és rendszerezése nem triviális feladat: a fejlesztőknek egyszerre kell megérteniük az operációs rendszer belső működését, a fenyegetési modelleket és a rendelkezésre álló védelmi eszközkészletet. A dolgozat ehhez kíván kapaszkodót nyújtani, áttekinthető szerkezetben bemutatva a releváns megoldásokat és azok gyakorlati alkalmazását.
